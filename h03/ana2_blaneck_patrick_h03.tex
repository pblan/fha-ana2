\documentclass[answers]{exam}

\usepackage[ngerman]{babel}
\usepackage{amsmath,amsthm,amsfonts,stmaryrd,amssymb,mathtools}
\usepackage{xcolor,soul}
\usepackage{polynom}
\usepackage{tikz}
\usepackage{footnote}
\usepackage{array}   % for \newcolumntype macro
\usepackage{pgfplots}
\usepgfplotslibrary{fillbetween}

\newcolumntype{L}{>{$}l<{$}} % math-mode version of "l" column type
\newcolumntype{R}{>{$}r<{$}} % math-mode version of "r" column type
\newcolumntype{C}{>{$}c<{$}} % math-mode version of "c" column type
\newcolumntype{P}{>{$}p<{$}} % math-mode version of "l" column type

\renewcommand*\env@matrix[1][*\c@MaxMatrixCols c]{%
  \hskip -\arraycolsep
  \let\@ifnextchar\new@ifnextchar
  \array{#1}}

\newcommand{\Rnum}[1]{\uppercase\expandafter{\romannumeral #1\relax}}

\newcommand{\abs}[1]{\left| #1 \right|}
\newcommand{\cis}[1]{\left( \cos\left( #1 \right) + i \sin\left( #1 \right) \right)}
\newcommand{\sgn}{\text{sgn}} % Signum-Funktion
\newcommand{\diff}{\mathrm{d}} % Differentialquotienten d
\newcommand{\df}{\mathrm{d}f} % dx
\newcommand{\dx}{\mathrm{d}x} % dx
\newcommand{\du}{\mathrm{d}u} % du
\newcommand{\dv}{\mathrm{d}v} % dv
\newcommand{\dw}{\mathrm{d}w} % dw
\newcommand{\dt}{\mathrm{d}t} % dt
\newcommand{\dn}{\mathrm{d}n} % dn
\newcommand{\dy}{\mathrm{d}y} % dy
\newcommand{\dz}{\mathrm{d}z} % dz
\newcommand{\dudx}{\frac{\mathrm{d}u}{\mathrm{d}x}} % du/dx
\newcommand{\dudn}{\frac{\mathrm{d}u}{\mathrm{d}n}} % du/dn
\newcommand{\dvdx}{\frac{\mathrm{d}v}{\mathrm{d}x}} % dv/dx
\newcommand{\dwdx}{\frac{\mathrm{d}w}{\mathrm{d}x}} % dw/dx
\newcommand{\dtdx}{\frac{\mathrm{d}t}{\mathrm{d}x}} % dt/dx
\newcommand{\ddx}{\frac{\mathrm{d}}{\mathrm{d}x}} % d/dx
\newcommand{\dFdx}{\frac{\mathrm{d}F}{\mathrm{d}x}} % dF/dx
\newcommand{\dfdx}{\frac{\mathrm{d}f}{\mathrm{d}x}}  % df/dx
\newcommand{\interval}[1]{\left[ #1 \right]}

\newcommand{\norm}[1]{\left\| #1 \right\|}
\newcommand{\scalarprod}[1]{\left\langle #1 \right\rangle}
\newcommand{\vektor}[1]{\begin{pmatrix*}[c] #1 \end{pmatrix*}}
%\newcommand{\vektor}[1]{\begin{pmatrix}[r] #1 \end{pmatrix}}
\renewcommand{\span}[1]{\operatorname{span}\left(#1\right)}

\newcommand{\Nplus}{\mathbb{N}^+}
\newcommand{\N}{\mathbb{N}}
\newcommand{\Z}{\mathbb{Z}}
\newcommand{\Rnonneg}{\mathbb{R}^+_0}
\newcommand{\R}{\mathbb{R}}
\newcommand{\C}{\mathbb{C}}
\newcommand{\bigo}{\mathcal{O}}
\newcommand{\Pot}{\mathcal{P}}

\DeclareMathOperator{\im}{im}
\DeclareMathOperator{\defect}{def}
\DeclareMathOperator{\rg}{rg}

\renewcommand{\solutiontitle}{\noindent\textbf{Lösung:}\par}


\makesavenoteenv{solution}
\lhead{Hausaufgabenblatt 03}
\rhead{Analysis 2}
\runningheadrule

\title{Analysis 2 \\ \large{Hausaufgabenblatt 03}}
\author{Patrick Gustav Blaneck}
\date{Abgabetermin: 18. April 2021}

\begin{document}
\maketitle
\begin{questions}
    \question
    Gegeben seien $f(x, y) = xy^2 - (2x+3y)^2$ und der Punkt $(x_0, y_0) = (2, -2)$.
    Berechnen Sie das vollständige Differential.
    \begin{solution}
        Das vollständige Differential von $z = f(x, y)$ ist gegeben mit
        $$
            \dz = f_x(x_0, y_0) \cdot \dx + f_y(x_0, y_0) \cdot \dy = \scalarprod{\nabla f(x_0, y_0), \vektor{\dx \\ \dy}}
        $$

        Es gilt für den Gradienten $\nabla f(x_0, y_0)$
        $$
            \nabla f(x_0, y_0) = \vektor{f_x(x_0, y_0) \\ f_y(x_0, y_0)} = \vektor{y_0^2 - 4(2x_0+3y_0) \\ 2x_0y_0 - 6(2x_0+3y_0)}
        $$

        Damit erhalten wir insgesamt das vollständige Differential am Punkt $(x_0, y_0) = (2, -2)$ mit
        $$
            \dz = f_x(2, -2) \cdot \dx + f_y(2, -2) \cdot \dy = 12\dx + 4\dy
        $$\qed
    \end{solution}

    \newpage

    \question
    Bei der Berechnung einer Fläche $f(x, y) = 5x^2 \cdot y$ werde ein relativer Messfehler von $10\%$ in $x$ und $3\%$ in $y$ gemacht.
    Wie ist der relative Fehler des Ergebnisses?
    \begin{solution}
        $z := f(x, y) = 5x^2 \cdot y \qquad \left( c \cdot x^a \cdot y^b\right)$

        Es ist der relative Fehler gegeben mit
        $$
            \frac{\varDelta z}{z} \leq a \cdot \abs{\frac{\varDelta x}{x}} + b \cdot \abs{\frac{\varDelta y}{y}} = 2 \cdot 10\% + 1\cdot 3\% = 23\%
        $$\qed
    \end{solution}

    \newpage

    \question
    Differenzieren Sie die gegebene Funktion $f(x,y)$ nach dem Parameter $t$ (längst der Kurve $C$)
    $$
        f(x, y) = x^2 + 2xy + 2y \qquad C : \vec{g}(t) = \vektor{x(t) \\ y(t)} = \vektor{t^2 \\ -2t}
    $$

    \begin{parts}
        \part
        unter Verwendung der Kettenregel,
        \begin{solution}
            $z := f(x, y)$
            $$
                \begin{aligned}
                                 & \frac{\dz}{\dt} &  & = \frac{\partial z}{\partial x} \frac{\dx}{\dt} + \frac{\partial z}{\partial y} \frac{\dy}{\dt} \\
                    \equiv \quad & \frac{\dz}{\dt} &  & =  \left(2x + 2y\right) \cdot 2t + \left(2x + 2\right) \cdot (-2)                               \\
                    \equiv \quad & \frac{\dz}{\dt} &  & =  \left(2t^2 -4t\right) \cdot 2t + \left(2t^2 + 2\right) \cdot (-2)                            \\
                    \equiv \quad & \frac{\dz}{\dt} &  & =       4t^3 -12t^2 -4                                                                          \\
                \end{aligned}
            $$\qed
        \end{solution}

        \part nach Einsetzen der beiden Parametergleichungen in die Funktionsgleichung.
        \begin{solution}
            $$
                f(x, y) = x^2 + 2xy + 2y \implies f(t) = t^4 - 4t^3 - 4t
            $$

            $$
                \frac{\df}{\dt} = 4t^3 -12t^2 -4
            $$\qed
        \end{solution}
    \end{parts}

    \newpage

    \question
    Berechnen Sie den Gradient der Funktion:
    $$
        f(r, s, t) = r \cdot \cos(t + s^2)
    $$
    \begin{solution}
        $$
            \nabla f(r, s, t) = \vektor{f_r(r, s, t) \\ f_s(r, s, t) \\ f_t(r, s, t)} = \vektor{\cos(t + s^2) \\ -2rs \cdot \sin(t + s^2) \\ -r \cdot \sin(t + s^2)}
        $$\qed
    \end{solution}

    \newpage

    \question
    Untersuchen Sie die folgenden Funktionen auf lokale Extrema und Sattelpunkte:
    \begin{parts}
        \part
        $g(x, y) = (x^2 + y)^2 + 4xy - x$
        \begin{solution}
            Wir berechnen zuerst die potentiellen Kandidaten.
            Für diese muss gelten
            $$
                \begin{aligned}
                    \nabla f(x, y) & = \vektor{f_x(x, y)                 \\ f_y(x, y)} && = \vec{0} \\
                    \implies \quad & \vektor{4x(x^2 + y) + 4y - 1        \\ 2(x^2 + y) + 4x} && = \vektor{0 \\ 0} \\
                    \implies \quad & \vektor{x(x^2 + y)+ y - \frac{1}{4} \\ x^2 + 2x + y } && = \vektor{0 \\ 0} \\
                    \implies \quad & \vektor{x(x^2 + y)+ y - \frac{1}{4} \\ -x^2 - 2x} && = \vektor{0 \\ y} \\
                \end{aligned}
            $$

            Wir haben offensichtlich zwei Gleichungen gegeben.

            Setzen wir die \Rnum{2}. Gleichung in die \Rnum{1}. Gleichung ein, erhalten wir
            $$
                x(x^2 + y) + y - \frac{1}{4} = x(x^2 -x^2 - 2x) -x^2 - 2x - \frac{1}{4} = \ldots = x^2 + \frac{2}{3}x + \frac{1}{12} = 0
            $$

            $$
                \begin{aligned}
                    x^2 + \frac{2}{3}x + \frac{1}{12} = 0 \implies x_{1, 2} = -\frac{1}{3} \pm \sqrt{\frac{1}{9} - \frac{1}{12}} \iff x_{1, 2} = \frac{-2 \pm 1}{6}
                \end{aligned}
            $$

            Setzen wir diese Ergebnisse dann wieder in Gleichung \Rnum{2} ein, erhalten wir:
            $$
                \begin{aligned}
                    y_1 & = -x_1^2 -2x_1 = -\left(-\frac{1}{2}\right)^2 -2 \cdot \left(-\frac{1}{2}\right) =-\frac{1}{4} + 1 = \frac{3}{4}              \\
                    y_2 & = -x_2^2 -2x_2 = -\left(-\frac{1}{6}\right)^2 -2 \cdot \left(-\frac{1}{6}\right) =-\frac{1}{36} + \frac{1}{3} = \frac{11}{36}
                \end{aligned}
            $$

            Unsere Kandidatentupel sind dann gegeben mit $(x_1, y_1) = \left(-\frac{1}{2}, \frac{3}{4}\right)$ und $(x_2, y_2) = \left(-\frac{1}{6}, \frac{11}{36}\right)$.

            Wir bilden nun die Hesse-Matrix:

            $$
                \begin{aligned}
                    f_x    & = 4x^3 + 4xy + 4y - 1 \\
                    f_{xx} & = 12x^2 + 4y          \\
                    f_{xy} & = 4x + 4              \\
                    f_y    & = 2x^2 + 2y + 4x      \\
                    f_{yy} & = 2
                \end{aligned}
            $$

            $$
                H = \vektor{f_{xx} & f_{xy} \\ f_{xy} & f_{yy}} = \vektor{12x^2 + 4y & 4x + 4 \\ 4x + 4 & 2}
            $$

            Für den Punkt $(x_1, y_1) = \left(-\frac{1}{2}, \frac{3}{4}\right)$ gilt damit:
            $$
                H_1 = \vektor{12 \cdot \left(-\frac{1}{2}\right)^2 + 4 \cdot \frac{3}{4} & 4 \cdot \left(-\frac{1}{2}\right) + 4 \\ 4 \cdot \left(-\frac{1}{2}\right) + 4 & 2} = \vektor{6 & 2 \\ 2 & 2}
            $$

            Offensichtlich ist $H_1$ positiv definit $\implies$ Minimum.

            Für den Punkt $(x_2, y_2) = \left(-\frac{1}{6}, \frac{11}{36}\right)$ gilt:
            $$
                H_2 = \vektor{12 \cdot \left(-\frac{1}{6}\right)^2 + 4 \cdot \frac{11}{36} & 4 \cdot \left(-\frac{1}{6}\right) + 4 \\ 4 \cdot \left(-\frac{1}{6}\right) + 4 & 2} = \vektor{\frac{14}{9} & \frac{10}{3} \\ \frac{10}{3} & 2}
            $$

            $H_2$ ist indefinit $\implies$ Sattelpunkt.\qed
        \end{solution}

        \part
        $v(x, y, z) = xy - z^4 -2(x^2 + y^2 -z^2)$
        \begin{solution}

            $v(x, y, z) = xy - z^4 -2(x^2 + y^2 -z^2) = xy - z^4 -2x^2 -2y^2 + 2z^2$

            Wir berechnen zuerst die potentiellen Kandidaten.
            Für diese muss gelten
            $$
                \begin{aligned}
                    \nabla v(x, y, z) & = \vektor{v_x(x, y, z)      \\ v_y(x, y, z)\\ v_z(x, y, z)} && = \vec{0} \\
                    \implies \quad    & \vektor{              y -4x \\ x -4y \\ -4z^3 +4z}              &  & = \vektor{0 \\ 0 \\ 0} \\
                    \implies \quad    & \vektor{              y     \\ x \\ -z^3 +z}              &  & = \vektor{4x \\ 4y \\ 0} \\
                    \implies \quad    & \vektor{              y     \\ x \\ -z(z^2 +1)}              &  & = \vektor{4x \\ 4y \\ 0} \\
                \end{aligned}
            $$
            Wir haben offensichtlich drei Gleichungen gegeben.

            Wir erkennen aus \Rnum{3} direkt, dass $z \in \{-1, 0, 1\}$ gelten muss und aus \Rnum{2} und \Rnum{1}, dass $x = y = 0$.

            Damit erhalten wir die drei Kandidatentupel:
            \begin{itemize}
                \item $(x_1, y_1, z_1) = (0, 0, -1)$,
                \item $(x_2, y_2, z_2) = (0, 0, 0)$,
                \item $(x_3, y_3, z_3) = (0, 0, 1)$.
            \end{itemize}

            Wir bilden nun die Hesse-Matrix:
            $$
                \begin{aligned}
                    f_{xx} = -4 \qquad         & f_{xy} = 1          &  & f_{xz} = 0         \\
                    f_{yx} = f_{xy} = 1 \qquad & f_{yy} = -4         &  & f_{yz} = 0         \\
                    f_{zx} = f_{xz} = 0 \qquad & f_{zy} = f_{yz} = 0 &  & f_{zz} = 12z^2 + 4
                \end{aligned}
            $$

            $$
                H = \vektor{f_{xx} & f_{xy} & f_{xz} \\ f_{yx} & f_{yy} & f_{yz} \\ f_{zx} & f_{zy} & f_{zz}} = \vektor{-4 & 1 & 0 \\ 1 & -4 & 0 \\ 0 & 0 & -12z^2 + 4}
            $$

            Es gilt für die Unterdeterminanten:
            $$
                \begin{aligned}
                    \det H_1 & = -4                                                                 \\
                    \det H_2 & = 15                                                                 \\
                    \det H   & = (-12z^2 + 4) \cdot \det H_2 = (-12z^2 + 4) \cdot 15 = -180z^2 + 60
                \end{aligned}
            $$

            Es gilt weiterhin
            $$
                \det H \leq 0 \iff -180z^2 + 60 \leq 0 \iff z^2 \leq \frac{1}{3}
            $$

            Damit ist die Hesse-Matrix für alle $\abs{z} \leq \frac{1}{9}$ indefinit und sonst negativ definit.

            Damit sind die Kandidatentupel $(x_1, y_1, z_1) = (0, 0, -1)$ und $(x_3, y_3, z_3) = (0, 0, 1)$ Maxima und $(x_2, y_2, z_2) = (0, 0, 0)$ ein Sattelpunkt. \qed
        \end{solution}
    \end{parts}
\end{questions}
\end{document}