\documentclass[answers]{exam}

\usepackage[ngerman]{babel}
\usepackage{amsmath,amsthm,amsfonts,stmaryrd,amssymb,mathtools}
\usepackage{xcolor,soul}
\usepackage{polynom}
\usepackage{tikz}
\usepackage{footnote}
\usepackage{array}   % for \newcolumntype macro
\usepackage{pgfplots}
\usepgfplotslibrary{fillbetween}

\newcolumntype{L}{>{$}l<{$}} % math-mode version of "l" column type
\newcolumntype{R}{>{$}r<{$}} % math-mode version of "r" column type
\newcolumntype{C}{>{$}c<{$}} % math-mode version of "c" column type
\newcolumntype{P}{>{$}p<{$}} % math-mode version of "l" column type

\renewcommand*\env@matrix[1][*\c@MaxMatrixCols c]{%
  \hskip -\arraycolsep
  \let\@ifnextchar\new@ifnextchar
  \array{#1}}

\newcommand{\abs}[1]{\left| #1 \right|}
\newcommand{\cis}[1]{\left( \cos\left( #1 \right) + i \sin\left( #1 \right) \right)}
\newcommand{\sgn}{\text{sgn}} % Signum-Funktion
\newcommand{\diff}{\mathrm{d}} % Differentialquotienten d
\newcommand{\dx}{~\mathrm{d}x} % dx
\newcommand{\du}{~\mathrm{d}u} % du
\newcommand{\dv}{~\mathrm{d}v} % dv
\newcommand{\dw}{~\mathrm{d}w} % dw
\newcommand{\dt}{~\mathrm{d}t} % dt
\newcommand{\dn}{~\mathrm{d}n} % dn
\newcommand{\dudx}{~\frac{\mathrm{d}u}{\mathrm{d}x}} % du/dx
\newcommand{\dudn}{~\frac{\mathrm{d}u}{\mathrm{d}n}} % du/dn
\newcommand{\dvdx}{~\frac{\mathrm{d}v}{\mathrm{d}x}} % dv/dx
\newcommand{\dwdx}{~\frac{\mathrm{d}w}{\mathrm{d}x}} % dw/dx
\newcommand{\dtdx}{~\frac{\mathrm{d}t}{\mathrm{d}x}} % dt/dx
\newcommand{\ddx}{\frac{\mathrm{d}}{\mathrm{d}x}} % d/dx
\newcommand{\dFdx}{\frac{\mathrm{d}F}{\mathrm{d}x}} % dF/dx
\newcommand{\dfdx}{\frac{\mathrm{d}f}{\mathrm{d}x}}  % df/dx
\newcommand{\interval}[1]{\left[ #1 \right]}

\newcommand{\norm}[1]{\left\| #1 \right\|}
\newcommand{\scalarprod}[1]{\left\langle #1 \right\rangle}
\newcommand{\vektor}[1]{\begin{pmatrix*}[c] #1 \end{pmatrix*}}
%\newcommand{\vektor}[1]{\begin{pmatrix}[r] #1 \end{pmatrix}}
\renewcommand{\span}[1]{\operatorname{span}\left(#1\right)}

\newcommand{\Nplus}{\mathbb{N}^+}
\newcommand{\N}{\mathbb{N}}
\newcommand{\Z}{\mathbb{Z}}
\newcommand{\Rnonneg}{\mathbb{R}^+_0}
\newcommand{\R}{\mathbb{R}}
\newcommand{\C}{\mathbb{C}}
\newcommand{\bigo}{\mathcal{O}}
\newcommand{\Pot}{\mathcal{P}}

\DeclareMathOperator{\im}{im}
\DeclareMathOperator{\defect}{def}
\DeclareMathOperator{\rg}{rg}

\renewcommand{\solutiontitle}{\noindent\textbf{Lösung:}\par}


\makesavenoteenv{solution}
\lhead{Hausaufgabenblatt 01}
\rhead{Analysis 2}
\runningheadrule

\title{Analysis 2 \\ \large{Hausaufgabenblatt 01}}
\author{Patrick Gustav Blaneck}
\date{Abgabetermin: 05. April 2021}

\begin{document}
\maketitle
\begin{questions}
    \question
    Berechnen Sie die Gradienten der folgenden Funktionen

    \begin{parts}
        \part
        $f(x, y) = xe^{x^2+y^2}$ an der Stelle $(1, 1)$.
        \begin{solution}
            Zuerst berechnen wie die partiellen Ableitungen $f_x$ und $f_y$:
            $$
                \begin{aligned}
                    f_x & = e^{x^2+y^2} + 2x^2e^{x^2+y^2} = e^{x^2+y^2}(1 + 2x^2) \\
                    f_y & = 2xye^{x^2+y^2}
                \end{aligned}
            $$

            Damit erhalten wir dann den Gradienten $\nabla f$ an der Stelle $(1, 1)$ mit:
            $$
                \nabla f(1, 1) = \vektor{f_x(1, 1) \\ f_y(1, 1)}
                = \vektor{3e^2                                               \\ 2e^2}
            $$\qed
        \end{solution}

        \part
        $f(x, y, z) = x^3 + y^2 + z$ an der Stelle $(1, 2, 3)$.
        \begin{solution}
            Zuerst berechnen wie die partiellen Ableitungen $f_x$, $f_y$ und $f_z$:
            $$
                \begin{aligned}
                    f_x & = 3x^2 \\
                    f_y & = 2y   \\
                    f_z & = 1
                \end{aligned}
            $$

            Damit erhalten wir dann den Gradienten $\nabla f$ an der Stelle $(1, 2, 3)$ mit:
            $$
                \nabla f(1, 2, 3) = \vektor{f_x(1, 2, 3) \\ f_y(1, 2, 3) \\ f_z(1, 2, 3)}
                = \vektor{3 \\ 4 \\ 1}
            $$\qed
        \end{solution}
    \end{parts}

    \newpage

    \question
    Gegeben sei die Funktion
    $$
        f(x, y) = (x^2 + y^2 -2)^2
    $$
    Geben Sie die Tangentialebene in den folgenden Punkten $(x_0, y_0)$ an:
    \begin{parts}
        \part
        $(1,0)$
        \begin{solution}
            Zuerst berechnen wie die partiellen Ableitungen $f_x$ und $f_y$:
            $$
                \begin{aligned}
                    f_x & = 4x(x^2 + y^2 -2) \\
                    f_y & = 4y(x^2 + y^2 -2)
                \end{aligned}
            $$
            $$
                z_0 = f(x_0, y_0) = f(1, 0) = 1
            $$
            Damit ergibt sich dann die Tangentialebene von $f$ am Punkt $(1, 0)$ mit:
            $$
                \begin{aligned}
                    E     & = \vektor{x_0 \\ y_0 \\ z_0} + \lambda \vektor{1 \\ 0 \\ f_x(x_0, y_0)} + \mu \vektor{0 \\ 1 \\ f_y(x_0, y_0)} \\
                    \quad & = \vektor{1   \\ 0 \\ 1} + \lambda \vektor{1 \\ 0 \\ -4} + \mu \vektor{0 \\ 1 \\ 0} \quad \lambda, \mu \in \R
                \end{aligned}
            $$\qed
        \end{solution}

        \part
        $(0,2)$
        \begin{solution}
            Wir haben die partiellen Ableitungen bereits in Aufgabenteil (a) berechnet.
            $$
                z_0 = f(x_0, y_0) = f(0, 2) = 4
            $$
            Damit ergibt sich dann die Tangentialebene von $f$ am Punkt $(0, 2)$ mit:
            $$
                \begin{aligned}
                    E     & = \vektor{x_0 \\ y_0 \\ z_0} + \lambda \vektor{1 \\ 0 \\ f_x(x_0, y_0)} + \mu \vektor{0 \\ 1 \\ f_y(x_0, y_0)} \\
                    \quad & = \vektor{0   \\ 2 \\ 4} + \lambda \vektor{1 \\ 0 \\ 0} + \mu \vektor{0 \\ 1 \\ 16}  \quad \lambda, \mu \in \R
                \end{aligned}
            $$\qed

        \end{solution}
    \end{parts}

    \newpage

    \question
    Berechnen Sie die Richtungsableitung der Funktion
    $$
        f(x, y) = x^2y - y^3x + 1
    $$
    im Punkt
    $$
        (x_0, y_0) = (1, 2)
    $$
    in Richtung des Vektors
    $$
        \vec{w} = \vektor{3 \\ 2}
    $$
    \begin{solution}
        Die Richtungsableitung von $f$ im Punkt $(x_0, y_0)$ in Richtung $\vec{v}$ ($\norm{\vec{v}} = 1$) ist gegeben mit
        $$
            \begin{aligned}
                D_{\vec{v}}(f) & = \scalarprod{\nabla f(x_0, y_0), \vec{v}}
            \end{aligned}
        $$

        Zuerst berechnen wir die partiellen Ableitungen $f_x$ und $f_y$:
        $$
            \begin{aligned}
                f_x = 2xy - y^3   & \implies f_x(x_0, y_0) = f_x(1, 2) = -4  \\
                f_x = x^2 - 3y^2x & \implies f_y(x_0, y_0) = f_y(1, 2) = -11
            \end{aligned}
        $$

        Sei nun $\vec{v} = \frac{\vec{w}}{\norm{\vec{w}}}$:
        $$
            \vec{v} = \frac{\vec{w}}{\norm{\vec{w}}} = \frac{1}{\sqrt{13}} \vektor{3  \\ 2}
        $$
        Damit können wir nun die Richtungsableitung wie folgt bilden:
        $$
            \begin{aligned}
                D_{\vec{v}}(f) & = \scalarprod{\nabla f(x_0, y_0), \vec{v}}    \\
                               & = \scalarprod{\vektor{-4                      \\ -11},  \frac{1}{\sqrt{13}} \vektor{3  \\ 2}} \\
                               & =  \frac{1}{\sqrt{13}} \scalarprod{\vektor{-4 \\ -11}, \vektor{3  \\ 2}} \\
                               & =  \frac{-12 -22}{\sqrt{13}}                  \\
                               & =  -\frac{34}{\sqrt{13}}
            \end{aligned}
        $$\qed
    \end{solution}

    \newpage

    \question
    Eine Funktion $f$ habe in Richtung des Vektors $\vec{v}_1 = \frac{1}{5}\vektor{3 \\ 4}$ eine Steigung von $D_{\vec{v}_1}(f) = 20$ und in Richtung des Vektors $\vec{v}_2 = \frac{1}{2}\vektor{\sqrt{2} \\ \sqrt{2}}$ eine Steigung von $D_{\vec{v}_2}(f) = 15\sqrt{2}$.

    \begin{parts}
        \part
        Wie lautet der Gradient dieser Funktion?
        \begin{solution}
            Es sind folgende Gleichungen gegeben:
            $$
                \begin{aligned}
                                 & D_{\vec{v}_1}(f)  = \scalarprod{\nabla f(x_0, y_0), \vec{v}_1}           \\
                    \equiv \quad & 20                = \scalarprod{\nabla f(x_0, y_0), \frac{1}{5}\vektor{3 \\ 4}} \\
                    \equiv \quad & 20                = \frac{1}{5}\scalarprod{\nabla f(x_0, y_0), \vektor{3 \\ 4}} \\
                    \equiv \quad & 100               = \scalarprod{\nabla f(x_0, y_0), \vektor{3            \\ 4}} \\
                    \equiv \quad & 100               = 3f_x(x_0, y_0) + 4f_y(x_0, y_0) \qquad (1)
                \end{aligned}
            $$
            und
            $$
                \begin{aligned}
                                 & D_{\vec{v}_2}(f) = \scalarprod{\nabla f(x_0, y_0), \vec{v}_2}                   \\
                    \equiv \quad & 15\sqrt{2}        = \scalarprod{\nabla f(x_0, y_0), \frac{1}{2}\vektor{\sqrt{2} \\ \sqrt{2}}} \\
                    \equiv \quad & 15\sqrt{2}        = \frac{1}{2}\scalarprod{\nabla f(x_0, y_0), \vektor{\sqrt{2} \\ \sqrt{2}}} \\
                    \equiv \quad & 30\sqrt{2}        = \scalarprod{\nabla f(x_0, y_0), \vektor{\sqrt{2}            \\ \sqrt{2}}} \\
                    \equiv \quad & 30\sqrt{2}        = \sqrt{2}f_x(x_0, y_0) + \sqrt{2}f_y(x_0, y_0)               \\
                    \equiv \quad & 30               = f_x(x_0, y_0) + f_y(x_0, y_0)                                \\
                    \equiv \quad & 90               = 3f_x(x_0, y_0) + 3f_y(x_0, y_0) \qquad (2)
                \end{aligned}
            $$
            Aus $(1)-(2)$ folgt:
            $$
                \begin{aligned}
                                 & 100 - 90  = 3f_x(x_0, y_0) + 4f_y(x_0, y_0) - (3f_x(x_0, y_0) + 3f_y(x_0, y_0)) \\
                    \equiv \quad & 10        = f_y(x_0, y_0)
                \end{aligned}
            $$

            Einsetzen in $(2)$ liefert dann:
            $$
                \begin{aligned}
                                 & 90  = 3f_x(x_0, y_0) + 3f_y(x_0, y_0) \\
                    \equiv \quad & 90  = 3f_x(x_0, y_0) + 30             \\
                    \equiv \quad & 20  = f_x(x_0, y_0)
                \end{aligned}
            $$

            Damit gilt für den Gradienten $\nabla f$
            $$
                \nabla f(x_0, y_0) = \vektor{f_x(x_0, y_0) \\ f_y(x_0, y_0)} = \vektor{20 \\ 10}
            $$\qed
        \end{solution}

        \newpage

        \part
        Welches ist die größtmögliche Steigung, die die Funktion in diesem unbekannten Punkt annehmen kann?
        \begin{solution}
            D maximale Steigung einer Funktion $f$ an einer Stelle $(x_0, y_0)$ ist gegeben mit
            $$
                \norm{\nabla f(x_0, y_0)} = \norm{\vektor{f_x(x_0, y_0) \\ f_y(x_0, y_0)}}
            $$

            Für unser Beispiel gilt damit
            $$
                \norm{\nabla f(x_0, y_0)} = \norm{\vektor{f_x(x_0, y_0) \\ f_y(x_0, y_0)}} = \norm{\vektor{20 \\ 10}} = 10\sqrt{5}
            $$\qed
        \end{solution}

        \part
        Welchen Winkel hat der Vektor $\vec{v}_1$ zum Gradienten?
        \begin{solution}
            $$
                \cos \varphi = \frac{\scalarprod{\vec{v}_1, \nabla f(x_0, y_0)}}{\norm{\vec{v}_1} \norm{\nabla f(x_0, y_0)}} = \frac{D_{\vec{v}_1}(f)}{\norm{\vec{v}_1}\norm{\nabla f(x_0, y_0)}} = \frac{D_{\vec{v}_1}(f)}{\norm{\nabla f(x_0, y_0)}} = \frac{20}{10\sqrt{5}} = \frac{2}{\sqrt{5}}
            $$

            Damit gilt $\varphi = \arccos \frac{2}{\sqrt{5}} \approx 0.4636$.\qed
        \end{solution}
    \end{parts}

    \newpage

    \question
    Sei $f : \R^2 \to \R$ mit $f(x, y) = (x^2 - y^3)^{\frac{1}{2}}$ und $(x_0, y_0) = (\sqrt{2}, 1)$.
    Bestimmen Sie
    \begin{parts}
        \part
        den Gradienten von $f$.
        \begin{solution}
            Wir berechnen zuerst die partiellen Ableitungen $f_x$ und $f_y$:
            $$
                \begin{aligned}
                    f_x & = \frac{x}{\sqrt{x^2-y^3}}      \\
                    f_y & = -\frac{3y^2}{2\sqrt{x^2-y^3}} \\
                \end{aligned}
            $$

            Der Gradient $\nabla f(x_0, y_0)$ ist dann gegeben mit
            $$
                \nabla f(x_0, y_0) = \vektor{f_x(x_0, y_0) \\ f_y(x_0, y_0)} = \vektor{\frac{x_0}{\sqrt{x_0^2-y_0^3}} \\ -\frac{3y_0^2}{2\sqrt{x_0^2-y_0^3}} }
            $$\qed
        \end{solution}

        \part
        die Gleichung der Tangentialebene an der Stelle $(x_0, y_0)$.
        \begin{solution}
            $$
                z_0 = f(x_0, y_0) = 1
            $$

            Damit ergibt sich dann die Tangentialebene von $f$ am Punkt $(\sqrt{2}, 1)$ mit:
            $$
                \begin{aligned}
                    E     & = \vektor{x_0      \\ y_0 \\ z_0} + \lambda \vektor{1 \\ 0 \\ f_x(x_0, y_0)} + \mu \vektor{0 \\ 1 \\ f_y(x_0, y_0)} \\
                    \quad & = \vektor{\sqrt{2} \\ 1 \\ 1} + \lambda \vektor{1 \\ 0 \\ \sqrt{2}} + \mu \vektor{0 \\ 1 \\ -\frac{3}{2}} \quad \lambda, \mu \in \R
                \end{aligned}
            $$\qed

        \end{solution}

        \part
        die Richtungsableitung an der Stelle $(x_0, y_0)$ in die Richtung $\vec{v} = \vektor{\frac{\sqrt{3}}{2} \\ \frac{1}{2}}$.
        \begin{solution}
            Die Richtungsableitung von $f$ im Punkt $(x_0, y_0)$ in Richtung $\vec{v}$ ($\norm{\vec{v}} = 1$) ist gegeben mit
            $$
                \begin{aligned}
                    D_{\vec{v}}(f) & = \scalarprod{\nabla f(x_0, y_0), \vec{v}}                                        \\
                                   & = \scalarprod{\vektor{\sqrt{2}                                                    \\ -\frac{3}{2}}, \vektor{\frac{\sqrt{3}}{2} \\ \frac{1}{2}}} \\
                                   & = \sqrt{2} \cdot \frac{\sqrt{3}}{2} + \left(-\frac{3}{2}\right) \cdot \frac{1}{2} \\
                                   & =  \sqrt{\frac{3}{2}}-\frac{3}{4}
                \end{aligned}
            $$\qed
        \end{solution}

        \part
        die Richtung, in der die Steigung im Punkt $(x_0, y_0)$ am größten ist und den Wert der größten Steigung.
        \begin{solution}
            Die Richtung, in der die Steigung am größten ist, entspricht dem Gradienten von $f$ an der Stelle $(x_0, y_0)$:
            $$
                \nabla f(x_0, y_0) = \vektor{f_x(x_0, y_0) \\ f_y(x_0, y_0)} = \vektor{f_x(\sqrt{2}, 1) \\ f_y(\sqrt{2}, 1)} = \vektor{\sqrt{2} \\ -\frac{3}{2}}
            $$

            Der Wert der Steigung ist dann
            $$
                \norm{\nabla f(x_0, y_0)} = \frac{\sqrt{17}}{2}
            $$\qed
        \end{solution}

        \part
        die Richtung, in der die Steigung im Punkt $(x_0, y_0)$ gleich Null ist.
        \begin{solution}
            Die Richtung, in der die Steigung im Punkt $(x_0, y_0)$ gleich Null ist, entspricht einem Vektor $\vec{w} \in \R^2$ mit $\vec{w} \perp \nabla f(x_0, y_0)$.

            Wir können ohne weiteren Beweis $\vec{w} = \vektor{\frac{3}{2} \\ \sqrt{2}}$ wählen.\qed
        \end{solution}
    \end{parts}
\end{questions}
\end{document}