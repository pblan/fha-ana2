\documentclass[answers]{exam}

\usepackage[ngerman]{babel}
\usepackage{amsmath,amsthm,amsfonts,stmaryrd,amssymb,mathtools}
\usepackage{xcolor,soul}
\usepackage{polynom}
\usepackage{tikz}
\usepackage{footnote}
\usepackage{array}   % for \newcolumntype macro
\usepackage{pgfplots}
\usepgfplotslibrary{fillbetween}

\newcolumntype{L}{>{$}l<{$}} % math-mode version of "l" column type
\newcolumntype{R}{>{$}r<{$}} % math-mode version of "r" column type
\newcolumntype{C}{>{$}c<{$}} % math-mode version of "c" column type
\newcolumntype{P}{>{$}p<{$}} % math-mode version of "l" column type

\newcommand{\abs}[1]{\left| #1 \right|}
\newcommand{\cis}[1]{\left( \cos\left( #1 \right) + i \sin\left( #1 \right) \right)}
\newcommand{\sgn}{\text{sgn}} % Signum-Funktion
\newcommand{\diff}{\mathrm{d}} % Differentialquotienten d
\newcommand{\dx}{~\mathrm{d}x} % dx
\newcommand{\du}{~\mathrm{d}u} % du
\newcommand{\dv}{~\mathrm{d}v} % dv
\newcommand{\dw}{~\mathrm{d}w} % dw
\newcommand{\dt}{~\mathrm{d}t} % dt
\newcommand{\dn}{~\mathrm{d}n} % dn
\newcommand{\dudx}{~\frac{\mathrm{d}u}{\mathrm{d}x}} % du/dx
\newcommand{\dudn}{~\frac{\mathrm{d}u}{\mathrm{d}n}} % du/dn
\newcommand{\dvdx}{~\frac{\mathrm{d}v}{\mathrm{d}x}} % dv/dx
\newcommand{\dwdx}{~\frac{\mathrm{d}w}{\mathrm{d}x}} % dw/dx
\newcommand{\dtdx}{~\frac{\mathrm{d}t}{\mathrm{d}x}} % dt/dx
\newcommand{\ddx}{\frac{\mathrm{d}}{\mathrm{d}x}} % d/dx
\newcommand{\dFdx}{\frac{\mathrm{d}F}{\mathrm{d}x}} % dF/dx
\newcommand{\dfdx}{\frac{\mathrm{d}f}{\mathrm{d}x}}  % df/dx
\newcommand{\interval}[1]{\left[ #1 \right]}

\newcommand{\norm}[1]{\left\| #1 \right\|}
\newcommand{\scalarprod}[1]{\left\langle #1 \right\rangle}
\newcommand{\vektor}[1]{\begin{pmatrix*}[r] #1 \end{pmatrix*}}
\renewcommand{\span}[1]{\operatorname{span}\left(#1\right)}

\renewcommand{\solutiontitle}{\noindent\textbf{Lösung:}\par}


\makesavenoteenv{solution}
\lhead{Hausaufgabenblatt 01}
\rhead{Analysis 2}
\runningheadrule

\title{Analysis 2 \\ \large{Hausaufgabenblatt 01}}
\author{Patrick Gustav Blaneck}
\date{Abgabetermin: 05. April 2021}

\begin{document}
\maketitle
\begin{questions}
    \question
    Berechnen Sie die 1. Ableitungen von
    \begin{parts}
        \part $f(x) = \sin(\cos(x))$
        \begin{solution}

        \end{solution}
        \part $g(x) = e^{x^3-\sin^2(x)}$
        \begin{solution}

        \end{solution}
        \part $h(x) = 3x^2 + 4x + (3-x^2)^4$
        \begin{solution}

        \end{solution}
        \part $k(x) = \frac{1}{\tan{\arcsin(x)}}$
        \begin{solution}

        \end{solution}
        \part $l(x) = e^{3\ln(x^2)}$
        \begin{solution}

        \end{solution}
        \part $m(x) = e^{x\cdot \sin^2(5\sqrt{x} + 17)} \cdot e^{x\cdot \cos^2(5\sqrt{x} + 17)}$
        \begin{solution}

        \end{solution}
    \end{parts}

    \newpage
    \question
    Berechnen Sie die Normen $\norm{\cdot}_1$, $\norm{\cdot}_2$ und $\norm{\cdot}_\infty$ für die folgenden Vektoren:
    \begin{parts}
        \part $\vektor{0\\2\\5\\1}, \vektor{9\\0\\6\\0}$
        \begin{solution}

        \end{solution}

        \part $\vektor{2\\3\\4}, \vektor{5\\0\\0}, \vektor{10\\2\\0}$
        \begin{solution}

        \end{solution}
    \end{parts}

    \newpage
    \question
    Berechnen Sie die partiellen Ableitungen erster Ordnung der folgenden Funktionen:
    \begin{parts}
        \part
        $f(x, y) = \arctan(\frac{x}{y})$
        \begin{solution}

        \end{solution}

        \part
        $f(x,y) = \tan{x^2 + y^2}$
        \begin{solution}

        \end{solution}

        \part
        $f(x,y) = \sqrt{9-x^2-y^2}$
        \begin{solution}

        \end{solution}
    \end{parts}

    \newpage
    \question
    Berechnen Sie die partiellen Ableitungen erster Ordnung der folgenden Funktionen:
    \begin{parts}
        \part
        $f(x, y) =2x^2-3xy-4y^2$
        \begin{solution}

        \end{solution}

        \part
        $f(x,y) = \frac{x^2}{y} + \frac{y^2}{x}, x \neq 0, y \neq 0$
        \begin{solution}

        \end{solution}

        \part
        $f(x,y) = \sin(2x + 3y)$
        \begin{solution}

        \end{solution}
    \end{parts}

    \newpage
    \question
    Lassen sich folgende Funktionen im Nullpunkt stetig ergänzen und, wenn ja, wie?
    \begin{parts}
        \part
        $f(x,y) = \frac{xy^2}{x^2 + y^8}$
        \begin{solution}

        \end{solution}

        \part
        $f(x,y) = \frac{x^3 + x^2 - y^4 + y^2}{x^2 + y^2}$
        \begin{solution}

        \end{solution}
    \end{parts}
\end{questions}
\end{document}