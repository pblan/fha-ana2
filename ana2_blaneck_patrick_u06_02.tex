\documentclass[answers]{exam}

\usepackage[ngerman]{babel}
\usepackage{amsmath,amsthm,amsfonts,stmaryrd,amssymb,mathtools}
\usepackage{xcolor,soul}
\usepackage{polynom}
\usepackage{nicefrac}
\usepackage{tikz}
\usepackage{footnote}
\usepackage{array}   % for \newcolumntype macro
\usepackage{pgfplots}
\usepgfplotslibrary{fillbetween}

\newcolumntype{L}{>{$}l<{$}} % math-mode version of "l" column type
\newcolumntype{R}{>{$}r<{$}} % math-mode version of "r" column type
\newcolumntype{C}{>{$}c<{$}} % math-mode version of "c" column type
\newcolumntype{P}{>{$}p<{$}} % math-mode version of "l" column type

\renewcommand*\env@matrix[1][*\c@MaxMatrixCols c]{%
  \hskip -\arraycolsep
  \let\@ifnextchar\new@ifnextchar
  \array{#1}}

\newcommand{\Rnum}[1]{\uppercase\expandafter{\romannumeral #1\relax}}

\newcommand{\abs}[1]{\left| #1 \right|}
\newcommand{\cis}[1]{\left( \cos\left( #1 \right) + i \sin\left( #1 \right) \right)}
\newcommand{\sgn}{\text{sgn}} % Signum-Funktion
\newcommand{\diff}{\mathrm{d}} % Differentialquotienten d
\renewcommand{\d}{\,\mathrm{d}}
\newcommand{\dudx}{\,\frac{\mathrm{d}u}{\mathrm{d}x}} % du/dx
\newcommand{\dudn}{\,\frac{\mathrm{d}u}{\mathrm{d}n}} % du/dn
\newcommand{\dvdx}{\,\frac{\mathrm{d}v}{\mathrm{d}x}} % dv/dx
\newcommand{\dwdx}{\,\frac{\mathrm{d}w}{\mathrm{d}x}} % dw/dx
\newcommand{\dtdx}{\,\frac{\mathrm{d}t}{\mathrm{d}x}} % dt/dx
\newcommand{\ddx}{\,\frac{\mathrm{d}}{\mathrm{d}x}} % d/dx
\newcommand{\dFdx}{\,\frac{\mathrm{d}F}{\mathrm{d}x}} % dF/dx
\newcommand{\dfdx}{\,\frac{\mathrm{d}f}{\mathrm{d}x}}  % df/dx
\newcommand{\interval}[1]{\left[ #1 \right]}

\newcommand{\norm}[1]{\left\| #1 \right\|}
\newcommand{\scalarprod}[1]{\left\langle #1 \right\rangle}
\newcommand{\vektor}[1]{\begin{pmatrix*}[c] #1 \end{pmatrix*}}
%\newcommand{\vektor}[1]{\begin{pmatrix}[r] #1 \end{pmatrix}}
\renewcommand{\span}[1]{\operatorname{span}\left(#1\right)}

\newcommand{\Nplus}{\mathbb{N}^+}
\newcommand{\N}{\mathbb{N}}
\newcommand{\Z}{\mathbb{Z}}
\newcommand{\Rnonneg}{\mathbb{R}^+_0}
\newcommand{\R}{\mathbb{R}}
\newcommand{\C}{\mathbb{C}}
\newcommand{\bigo}{\mathcal{O}}
\newcommand{\Pot}{\mathcal{P}}

\DeclareMathOperator{\im}{im}
\DeclareMathOperator{\defect}{def}
\DeclareMathOperator{\rg}{rg}
\DeclareMathOperator{\curl}{curl}

\renewcommand{\solutiontitle}{\noindent\textbf{Lösung:}\par}


\makesavenoteenv{solution}
\lhead{Hausaufgabenblatt 06}
\rhead{Analysis 2}
\runningheadrule

\title{Analysis 2 \\ \large{Hausaufgabenblatt 05}}
\author{Patrick Gustav Blaneck}
\date{Abgabetermin: 02. Mai 2021}

\begin{document}
\begin{questions}
    \setcounter{question}{1}
    \question
    Berechnen Sie folgendes Integral unter Verwendung der Polarkoordinaten
    $$
        \int_A \left(3\sqrt{x^2 + y^2} + 4\right)x \d A
    $$
    für den halben Kreisring $A$, dessen Mittelpunkt im Ursprung des kartesischen Koordinatensystems liegt, den inneren Radius $2$ und den äußeren Radius $3$ besitzt, und durch $x\geq 0$ bestimmt ist.
    \begin{solution}
        \begin{center}
            \begin{tikzpicture}[scale=1]
                \begin{axis}[
                        width=15cm,
                        xmin=0,xmax=3,
                        xtick distance=1,
                        xlabel = $x$,
                        ymin=-3.5,ymax=3.5,
                        ytick distance=1,
                        ylabel = $y$,
                        axis equal,
                        grid=both,
                        axis lines = middle,
                        disabledatascaling
                    ]
                    \draw [name path=A] (axis cs:0,-2) arc[start angle=-90, end angle=90, radius={transformdirectionx(2)}];
                    \draw [name path=B] (axis cs:0,-3) arc[start angle=-90, end angle=90, radius={transformdirectionx(3)}];
                    \tikzfillbetween[of=A and B]{blue, opacity=0.2};
                \end{axis}
            \end{tikzpicture}
        \end{center}

        Wir erkennen den Integrationsbereich:
        $$
            A = \left\{(r, \phi) \mid (2 \leq r \leq 3 ) \land \left(-\frac{\pi}{2} \leq \phi \leq \frac{\pi}{2}\right)\right\}
        $$

        Umwandeln des gegebenen Integrals in Polarkoordinaten ergibt:
        $$
            \int_A \left(3\sqrt{x^2 + y^2} + 4\right)x \d A = \int_A \left(3\sqrt{r^2} + 4\right)r\cos\phi \cdot r \d A = \int_A \left( 3r^3+4r^2 \right)\cos\phi \d A
        $$

        Einsetzen der Integralgrenzen:
        $$
            \begin{aligned}
                \int_A \left( 3r^3+4r^2 \right)\cos\phi \d A & = \int^{\frac{\pi}{2}}_{-\frac{\pi}{2}} \int^3_2 \left( 3r^3+4r^2 \right)\cos\phi \d r\d \phi                                                    \\
                                                             & = \int^{\frac{\pi}{2}}_{-\frac{\pi}{2}} \cos\phi \int^3_2 \left( 3r^3+4r^2 \right) \d r\d \phi                                                   \\
                                                             & = \int^{\frac{\pi}{2}}_{-\frac{\pi}{2}} \cos\phi \left[ \frac{3r^4}{4}+\frac{4r^3}{3} \right]^3_2 \d \phi                                        \\
                                                             & = \int^{\frac{\pi}{2}}_{-\frac{\pi}{2}} \cos\phi \left( \frac{243}{4} + \frac{108}{3} - \left(\frac{48}{4} + \frac{32}{3}\right) \right) \d \phi \\
                                                             & = \int^{\frac{\pi}{2}}_{-\frac{\pi}{2}} \cos\phi \cdot \frac{889}{12} \d \phi                                                                    \\
                                                             & = \frac{889}{12}\int^{\frac{\pi}{2}}_{-\frac{\pi}{2}} \cos\phi \d \phi                                                                           \\
                                                             & = \frac{889}{12} \left[\sin \phi\right]^{\frac{\pi}{2}}_{-\frac{\pi}{2}}                                                                         \\
                                                             & = \frac{889}{12} \left(1 + 1\right)                                                                                                              \\
                                                             & = \frac{889}{6}
            \end{aligned}
        $$\qed

    \end{solution}
\end{questions}
\end{document}