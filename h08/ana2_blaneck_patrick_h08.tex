\documentclass[answers]{exam}

\usepackage[ngerman]{babel}
\usepackage{amsmath,amsthm,amsfonts,stmaryrd,amssymb,mathtools}
\usepackage{xcolor,soul}
\usepackage{polynom}
\usepackage{nicefrac}
\usepackage{tikz}
\usepackage{footnote}
\usepackage{siunitx}
\usepackage{array}   % for \newcolumntype macro
\usepackage{pgfplots}
\usepgfplotslibrary{fillbetween}

\newcolumntype{L}{>{$}l<{$}} % math-mode version of "l" column type
\newcolumntype{R}{>{$}r<{$}} % math-mode version of "r" column type
\newcolumntype{C}{>{$}c<{$}} % math-mode version of "c" column type
\newcolumntype{P}{>{$}p<{$}} % math-mode version of "l" column type

\renewcommand*\env@matrix[1][*\c@MaxMatrixCols c]{%
  \hskip -\arraycolsep
  \let\@ifnextchar\new@ifnextchar
  \array{#1}}

\newcommand{\Rnum}[1]{\uppercase\expandafter{\romannumeral #1\relax}}

\newcommand{\abs}[1]{\left| #1 \right|}
\newcommand{\cis}[1]{\left( \cos\left( #1 \right) + i \sin\left( #1 \right) \right)}
\newcommand{\sgn}{\text{sgn}} % Signum-Funktion
\newcommand{\diff}{\mathrm{d}} % Differentialquotienten d
\renewcommand{\d}{\,\mathrm{d}}
\newcommand{\dudx}{\,\frac{\mathrm{d}u}{\mathrm{d}x}} % du/dx
\newcommand{\dudn}{\,\frac{\mathrm{d}u}{\mathrm{d}n}} % du/dn
\newcommand{\dvdx}{\,\frac{\mathrm{d}v}{\mathrm{d}x}} % dv/dx
\newcommand{\dwdx}{\,\frac{\mathrm{d}w}{\mathrm{d}x}} % dw/dx
\newcommand{\dtdx}{\,\frac{\mathrm{d}t}{\mathrm{d}x}} % dt/dx
\newcommand{\ddx}{\,\frac{\mathrm{d}}{\mathrm{d}x}} % d/dx
\newcommand{\dFdx}{\,\frac{\mathrm{d}F}{\mathrm{d}x}} % dF/dx
\newcommand{\dfdx}{\,\frac{\mathrm{d}f}{\mathrm{d}x}}  % df/dx
\newcommand{\interval}[1]{\left[ #1 \right]}

\newcommand{\norm}[1]{\left\| #1 \right\|}
\newcommand{\scalarprod}[1]{\left\langle #1 \right\rangle}
\newcommand{\vektor}[1]{\begin{pmatrix*}[c] #1 \end{pmatrix*}}
%\newcommand{\vektor}[1]{\begin{pmatrix}[r] #1 \end{pmatrix}}
\renewcommand{\span}[1]{\operatorname{span}\left(#1\right)}

\newcommand{\Nplus}{\mathbb{N}^+}
\newcommand{\N}{\mathbb{N}}
\newcommand{\Z}{\mathbb{Z}}
\newcommand{\Rnonneg}{\mathbb{R}^+_0}
\newcommand{\R}{\mathbb{R}}
\newcommand{\C}{\mathbb{C}}
\newcommand{\bigo}{\mathcal{O}}
\newcommand{\Pot}{\mathcal{P}}

\DeclareMathOperator{\im}{im}
\DeclareMathOperator{\defect}{def}
\DeclareMathOperator{\rg}{rg}
\DeclareMathOperator{\curl}{curl}

\renewcommand{\solutiontitle}{\noindent\textbf{Lösung:}\par}


\makesavenoteenv{solution}
\lhead{Hausaufgabenblatt 08}
\rhead{Analysis 2}
\runningheadrule

\title{Analysis 2 \\ \large{Hausaufgabenblatt 08}}
\author{Patrick Gustav Blaneck}
\date{Abgabetermin: 24. Mai 2021}

\begin{document}
\maketitle
\begin{questions}
    \question
    Bestimmen Sie die allgemeine Lösung der jeweiligen Differentialgleichung.
    \begin{parts}
        \part
        $y' - 4y = 14x + 2 - 12x^2$
        \begin{solution}
            Wir lösen zuerst die homogene Gleichung ($y'_h + ay_h= 0$):
            $$
                y_h'- 4y_h = 0 \implies y_h = ce^{-ax} = ce^{4x}
            $$

            Bestimmen der partikulären Lösung ($y_p = ax^2 + bx + c$):
            $$
                y_p = ax^2 + bx + c \implies y'_p = 2ax + b
            $$

            Wir setzen die partikuläre Lösung entsprechend ein:
            $$
                \begin{aligned}
                                 & y' - 4y                      &  & = 14x + 2 - 12x^2   \\
                    \equiv \quad & 2ax + b - 4( ax^2 + bx + c)  &  & = - 12x^2 + 14x + 2 \\
                    \equiv \quad & -4ax^2 + (2a-4b)x + (b - 4c) &  & = - 12x^2 + 14x + 2
                \end{aligned}
            $$
            Damit sehen wir auch direkt:
            $$
                a = 3 \quad \land \quad b = -2 \quad \land \quad c = -1,
            $$

            und schließlich:
            $$
                y = y_h + y_p \implies y = ce^{4x} + 3x^2 - 2x - 1.
            $$\qed
        \end{solution}

        \newpage
        \part
        $y' - 8y = -6e^{5x}$
        \begin{solution}
            Wir lösen zuerst die homogene Gleichung:
            $$
                y_h' - 8y_h' = 0 \implies y_h = ce^{8x}
            $$

            Bestimmen der partikulären Lösung ($y_p = c_0e^{5x}$):
            $$
                y_p = c_0e^{5x} \implies y_p' = 5ce^{5x}
            $$

            Wir setzen die partikuläre Lösung entsprechend ein:
            $$
                \begin{aligned}
                                 & y' - 8y                 &  & = -6e^{5x} \\
                    \equiv \quad & 5c_0e^{5x} - 8c_0e^{5x} &  & = -6e^{5x} \\
                    \equiv \quad & - 3c_0e^{5x}            &  & = -6e^{5x}
                \end{aligned}
            $$
            Damit sehen wir auch direkt:
            $$
                c_0 = 2,
            $$
            und schließlich:
            $$
                y = y_h + y_p \implies y = ce^{8x} + 2e^{5x}.
            $$\qed
        \end{solution}

        \part
        $y' + 2y = 8\sin(x) - \cos(x)$
        \begin{solution}
            Lösen der homogenen Gleichung:
            $$
                y_h' + 2y_h = 0 \implies y_h = ce^{-2x}
            $$

            Mit der partikulären Lösung ($y_p = c_0\sin(x) + c_1\cos(x)$)
            $$
                y_p = c_0\sin(x) + c_1\cos(x) \implies y_p' = c_0\cos(x) - c_1\sin(x)
            $$
            gilt dann:
            $$
                \begin{aligned}
                                   & y' + 2y                                              &  & = 8\sin(x) - \cos(x) \\
                    \equiv \quad   & c_0\cos(x) - c_1\sin(x) + 2(c_0\sin(x) + c_1\cos(x)) &  & = 8\sin(x) - \cos(x) \\
                    \equiv \quad   & (c_0 + 2c_1)\cos(x) + (2c_0 - c_1)\sin(x)            &  & = 8\sin(x) - \cos(x) \\
                    \implies \quad & c_0 + 2c_1 = -1 \quad \land \quad 2c_0 - c_1 = 8                               \\
                    \implies \quad & c_0 = 3 \quad \land \quad c_1 = -2
                \end{aligned}
            $$

            Und schlussendlich:
            $$
                y = y_h + y_p \implies y = ce^{-2x} + 3\sin(x) - 2\cos(x).
            $$\qed
        \end{solution}
    \end{parts}

    \newpage
    \question
    Berechnen Sie die allgemeine Lösung der linearen Differentialgleichungen
    \begin{parts}
        \part
        $y'- 4y = 15e^x$
        \begin{solution}
            Lösen der homogenen Gleichung:
            $$
                y_h'- 4y_h = 0 \implies y_h = ce^{4x}
            $$

            Mit der partikulären Lösung ($y_p = c_0e^{x}$)
            $$
                y_p = c_0e^x \implies y'_p = c_0e^x
            $$
            gilt dann:
            $$
                \begin{aligned}
                                   & y'- 4y          &  & = 15e^x \\
                    \equiv \quad   & c_0e^x- 4c_0e^x &  & = 15e^x \\
                    \equiv \quad   & -3c_0e^x        &  & = 15e^x \\
                    \implies \quad & c_0             &  & = -5
                \end{aligned}
            $$

            Und schlussendlich:
            $$
                y = y_h + y_p \implies y = ce^{4x} -5e^x
            $$\qed
        \end{solution}

        \part
        $y'- y = 9e^x$
        \begin{solution}
            Lösen der homogenen Gleichung:
            $$
                y_h'- y_h = 0 \implies y_h = ce^{x}
            $$

            Mit der partikulären Lösung ($y_p = c_0xe^{x}$)
            $$
                y_p = c_0xe^x \implies y'_p = c_0(e^x + xe^x)
            $$
            gilt dann:
            $$
                \begin{aligned}
                                   & y'- y                     &  & = 9e^x \\
                    \equiv \quad   & c_0(e^x + xe^x) - c_0xe^x &  & = 9e^x \\
                    \equiv \quad   & c_0e^x(1 + x - x)         &  & = 9e^x \\
                    \implies \quad & c_0                       &  & = 9
                \end{aligned}
            $$

            Und schlussendlich:
            $$
                y = y_h + y_p \implies y = ce^x + 9e^xx
            $$\qed
        \end{solution}
    \end{parts}

    \newpage
    \question
    Bestimmen Sie die allgemeine Lösung der folgenden DGL:
    $$
        x \cdot y'-2y = x^3 \cdot \sqrt{y}, \quad y(1) = 1
    $$
    \begin{solution}
        Wir haben eine Bernoulli-DGL 1. Ordnung ($y'(x) + f(x) \cdot y(x) = g(x) \cdot y^\alpha$) gegeben mit:
        $$
            x \cdot y'-2y = x^3 \cdot \sqrt{y} \quad \overset{x\neq 0}{\iff} \quad  y' - \frac{2}{x} \cdot y = x^2 \cdot \sqrt{y}
        $$
        Wir substituieren
        $$
            \begin{aligned}
                           & z = y^{1-\alpha} = \sqrt{y} &  & \implies z' = \frac{1}{2\sqrt{y}} \\
                \iff \quad & y = z^2                     &  & \implies y' = 2zz'
            \end{aligned}
        $$
        Daraus folgt:
        $$
            \begin{aligned}
                             & y' - \frac{2}{x} \cdot y     &  & = x^2 \cdot \sqrt{y}   \\
                \equiv \quad & 2zz' - \frac{2}{x} \cdot z^2 &  & = x^2 \cdot \sqrt{z^2} \\
                \equiv \quad & 2z' - \frac{2}{x} \cdot z    &  & = x^2                  \\
                \equiv \quad & z' - \frac{1}{x} \cdot z     &  & = \frac{x^2}{2}        \\
            \end{aligned}
        $$
        Lösen der homogenen Gleichung:
        $$
            \begin{aligned}
                             & z' - \frac{1}{x} \cdot z &  & = 0                       \\
                \equiv \quad & \frac{z'}{z}             &  & = \frac{1}{x}             \\
                \equiv \quad & \int \frac{1}{z} \d z    &  & = \int \frac{1}{x} \d x   \\
                \equiv \quad & \ln \abs{z} + c_1        &  & = \ln \abs{x} + c_2       \\
                \equiv \quad & \abs{z}                  &  & = e^{\abs{x} + c_2 - c_1} \\
                \equiv \quad & z                        &  & = cx
            \end{aligned}
        $$
        Bestimmen der partikulären Lösung:
        $$
            z = c(x) \cdot x \quad \implies \quad z' = c'(x)x + c(x)
        $$
        $$
            \begin{aligned}
                             & z' - \frac{1}{x} \cdot z                       &  & = \frac{x^2}{2}           \\
                \equiv \quad & c'(x)x + c(x) - \frac{1}{x} \cdot c(x) \cdot x &  & = \frac{x^2}{2}           \\
                \equiv \quad & c'(x)                                          &  & = \frac{x}{2}             \\
                \equiv \quad & \int 1 \d c                                    &  & = \frac{1}{2} \int x \d x \\
                \equiv \quad & c(x) + c_1                                     &  & = \frac{x^2}{4} + c_2     \\
                \equiv \quad & c(x)                                           &  & = \frac{x^2}{4} + c_0
            \end{aligned}
        $$
        \newpage
        Damit gilt insgesamt:
        $$
            z = c(x) \cdot x = \left(\frac{x^2}{4} + c_0\right)x = \frac{x^3}{4} + c_1x \implies y = z^2 = \left(\frac{x^3}{4} + c_1x\right)^2
        $$
        Mit dem Anfangswert $y(1) = 1$ folgt:
        $$
            y(1) = \left(\frac{1}{4} + c_1\right)^2 \implies c_1 \in \left\{\frac{3}{4}, -\frac{5}{4}\right\} \iff c_1 = -\frac{1}{4} \pm 1
        $$
        und damit schlussendlich
        $$
            y = z^2 = \left(\frac{x^3}{4} + \left(-\frac{1}{4} \pm 1\right)x\right)^2
        $$
        bzw.
        $$
            y_1 = \left(\frac{x^3}{4} + \frac{3x}{4}\right)^2 = \frac{x^2\left(x^2 + 3\right)^2}{16}
        $$
        und
        $$
            y_2 = \left(\frac{x^3}{4} - \frac{5x}{4}\right)^2 = \frac{x^2\left(x^2 -5\right)^2}{16}
        $$\qed
    \end{solution}

    \newpage
    \question
    Bestimmen Sie die Lösung der folgenden Differentialgleichung
    $$
        y' - 2y = (2\sin(x) + 5\cos(x)) \cdot e^{-3x}
    $$
    \begin{solution}
        Lösen der homogenen Gleichung:
        $$
            y_h' - 2y_h = 0 \quad \implies \quad y_h = ce^{2x}
        $$

        Mit der partikulären Lösung ($y_p = (c_1\sin(x) + c_2\cos(x))\cdot c_0e^{-3x}$)
        $$
            y_p = c_0e^{-3x}(c_1\sin(x) + c_2\cos(x)) \quad \implies \quad y_p' = c_0 e^{-3 x} ((c_1 - 3 c_2) \cos(x) + (-3 c_1 - c_2) \sin(x))
        $$
        gilt dann:
        $$
            \begin{aligned}
                             & y_p' - 2y_p                                                 &  & = (2\sin(x) + 5\cos(x)) e^{-3x} \\
                \equiv \quad & \ldots                                                      &  & = (2\sin(x) + 5\cos(x)) e^{-3x} \\
                \equiv \quad & c_0 e^{-3 x} ((c_1-5c_2)\cos(x) + (-5c_1 -c_2)\sin(x))      &  & = (2\sin(x) + 5\cos(x)) e^{-3x} \\
                \equiv \quad & c_0(c_1-5c_2) \cdot \cos(x) + c_0(-5c_1 -c_2) \cdot \sin(x) &  & = 2\sin(x) + 5\cos(x)           \\
            \end{aligned}
        $$

        Wir erhalten das LGS:
        $$
            \begin{cases}
                c_0(c_1 -5c_2)   & = 5 \\
                c_0(-5c_1 - c_2) & = 2
            \end{cases}
            \implies
            \ldots
            \implies
            \begin{cases}
                c_1 & = -\frac{5}{26c_0}  \\
                c_2 & = -\frac{27}{26c_0}
            \end{cases}
        $$

        Damit erhalten wir unsere partikuläre Lösung:
        $$
            y_p = c_0e^{-3x}(c_1\sin(x) + c_2\cos(x)) = -e^{-3x}\left(\frac{5}{26}\sin(x) + \frac{27}{26}\cos(x)\right)
        $$

        Damit gilt insgesamt:
        $$
            y = y_h + y_p = ce^{2x} - e^{-3x}\left(\frac{5}{26}\sin(x) + \frac{27}{26}\cos(x)\right)
        $$\qed
    \end{solution}

    \newpage
    \question
    Bestimmen Sie die allgemeine Lösung der Differentialgleichung
    $$
        y' = -y + x\cdot e^{-x} + 1
    $$
    \begin{solution}
        Es gilt ganz offensichtlich:
        $$
            y' = -y + x\cdot e^{-x} + 1 \quad \equiv \quad y'  + y = x\cdot e^{-x} + 1
        $$

        Lösen der homogenen Gleichung:
        $$
            y_h = ce^{-x}
        $$

        Wir müssen zwei partikuläre Lösungen erhalten:
        $$
            y_{p_1} = c_1 \cdot x^2 \cdot e^{-x} \quad \text{und} \quad y_{p_2} = c_2
        $$
        Es gilt:
        $$
            y_{p_1} = c_1 x^2  e^{-x} \quad \implies \quad y'_{p_1} = 2c_1xe^{-x} - c_1x^2e^{-x}
        $$

        $$
            \begin{aligned}
                               & y_{p_1} + y'_{p_1}                          &  & = xe^{-x} \\
                \equiv \quad   & c_1 x^2 e^{-x} + 2c_1xe^{-x} - c_1x^2e^{-x} &  & = xe^{-x} \\
                \equiv \quad   & 2c_1xe^{-x}                                 &  & = xe^{-x} \\
                \implies \quad & c_1 = \frac{1}{2}
            \end{aligned}
        $$

        und
        $$
            y_{p_2} = c_2 \quad \implies \quad y'_{p_2} = 0
        $$

        $$
            \begin{aligned}
                               & y_{p_2} + y'_{p_2} &  & = 1 \\
                \equiv \quad   & c_2 + 0            &  & = 1 \\
                \implies \quad & c_2 = 1
            \end{aligned}
        $$

        Damit gilt insgesamt:
        $$
            y = y_h + y_{p_1} + y_{p_2} = ce^{-x} + \frac{x^2e^{-x}}{2} + 1
        $$\qed
    \end{solution}
\end{questions}
\end{document}