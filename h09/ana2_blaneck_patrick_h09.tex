\documentclass[answers]{exam}

\usepackage[ngerman]{babel}
\usepackage{amsmath,amsthm,amsfonts,stmaryrd,amssymb,mathtools}
\usepackage{xcolor,soul}
\usepackage{polynom}
\usepackage{nicefrac}
\usepackage{tikz}
\usepackage{footnote}
\usepackage{siunitx}
\usepackage{array}   % for \newcolumntype macro
\usepackage{pgfplots}
\usepgfplotslibrary{fillbetween}

\newcolumntype{L}{>{$}l<{$}} % math-mode version of "l" column type
\newcolumntype{R}{>{$}r<{$}} % math-mode version of "r" column type
\newcolumntype{C}{>{$}c<{$}} % math-mode version of "c" column type
\newcolumntype{P}{>{$}p<{$}} % math-mode version of "l" column type

\renewcommand*\env@matrix[1][*\c@MaxMatrixCols c]{%
  \hskip -\arraycolsep
  \let\@ifnextchar\new@ifnextchar
  \array{#1}}

\newcommand{\Rnum}[1]{\uppercase\expandafter{\romannumeral #1\relax}}

\newcommand{\seq}{\overset{!}{=}}
\newcommand{\abs}[1]{\left| #1 \right|}
\newcommand{\cis}[1]{\left( \cos\left( #1 \right) + i \sin\left( #1 \right) \right)}
\newcommand{\sgn}{\text{sgn}} % Signum-Funktion
\newcommand{\diff}{\mathrm{d}} % Differentialquotienten d
\renewcommand{\d}{\,\mathrm{d}}
\newcommand{\dudx}{\,\frac{\mathrm{d}u}{\mathrm{d}x}} % du/dx
\newcommand{\dudn}{\,\frac{\mathrm{d}u}{\mathrm{d}n}} % du/dn
\newcommand{\dvdx}{\,\frac{\mathrm{d}v}{\mathrm{d}x}} % dv/dx
\newcommand{\dwdx}{\,\frac{\mathrm{d}w}{\mathrm{d}x}} % dw/dx
\newcommand{\dtdx}{\,\frac{\mathrm{d}t}{\mathrm{d}x}} % dt/dx
\newcommand{\ddx}{\,\frac{\mathrm{d}}{\mathrm{d}x}} % d/dx
\newcommand{\dFdx}{\,\frac{\mathrm{d}F}{\mathrm{d}x}} % dF/dx
\newcommand{\dfdx}{\,\frac{\mathrm{d}f}{\mathrm{d}x}}  % df/dx
\newcommand{\interval}[1]{\left[ #1 \right]}

\newcommand{\norm}[1]{\left\| #1 \right\|}
\newcommand{\scalarprod}[1]{\left\langle #1 \right\rangle}
\newcommand{\vektor}[1]{\begin{pmatrix*}[c] #1 \end{pmatrix*}}
%\newcommand{\vektor}[1]{\begin{pmatrix}[r] #1 \end{pmatrix}}
\renewcommand{\span}[1]{\operatorname{span}\left(#1\right)}

\newcommand{\Nplus}{\mathbb{N}^+}
\newcommand{\N}{\mathbb{N}}
\newcommand{\Z}{\mathbb{Z}}
\newcommand{\Rnonneg}{\mathbb{R}^+_0}
\newcommand{\R}{\mathbb{R}}
\newcommand{\C}{\mathbb{C}}
\newcommand{\bigo}{\mathcal{O}}
\newcommand{\Pot}{\mathcal{P}}

\DeclareMathOperator{\im}{im}
\DeclareMathOperator{\defect}{def}
\DeclareMathOperator{\rg}{rg}
\DeclareMathOperator{\curl}{curl}

\renewcommand{\solutiontitle}{\noindent\textbf{Lösung:}\par}


\makesavenoteenv{solution}
\lhead{Hausaufgabenblatt 09}
\rhead{Analysis 2}
\runningheadrule

\title{Analysis 2 \\ \large{Hausaufgabenblatt 09}}
\author{Patrick Gustav Blaneck}
\date{Abgabetermin: 30. Mai 2021}

\begin{document}
\maketitle
\begin{questions}
    \question
    Lösen Sie die Differentialgleichung
    $$
        (3xy + 2y^2) + (x^2 + 2xy) y' = 0
    $$
    \begin{solution}
        Es gilt:
        $$
            (3xy + 2y^2) + (x^2 + 2xy) y' = 0 \quad \iff \quad \underbrace{(3xy + 2y^2)}_{p(x, y)}\d x + \underbrace{(x^2 + 2xy)}_{q(x, y)}\d y = 0
        $$
        Integrabilitätsbedingung:
        $$
            p_y = 3x + 4y \quad \neq \quad 2x + 2y = q_x \quad \lightning
        $$

        Integrierenden Faktor (Euler-Multiplikator) $\mu(x) = e^{\int m(x) \d x}$ bzw. $\mu(y) = e^{\int m(y) \d y}$ bestimmen:
        \begin{itemize}
            \item Untersuchung, ob $\mu$ nur von $y$ abhängt:
                  $$
                      m = \frac{q_x - p_y}{p} = \frac{2x + 2y - \left(3x + 4y\right)}{3xy + 2y^2} = \frac{-x - 2y}{3xy + 2y^2} \quad \lightning
                  $$
            \item Untersuchung, ob $\mu$ nur von $x$ abhängt:
                  $$
                      m = \frac{p_y - q_x}{q} = \frac{3x + 4y - (2x + 2y)}{x^2 + 2xy} = \frac{x + 2y}{x^2 + 2xy} = \frac{x+2y}{x(x+2y)} = \frac{1}{x} \quad \checkmark
                  $$
        \end{itemize}

        Damit erhalten wir den integrierenden Faktor mit:
        $$
            \mu(x) = e^{\int \frac{1}{x} \d x} = cx \qquad (\text{sei} \ c = 1)
        $$
        Einsetzen in die DGL:
        $$
            \underbrace{\left(3x^2y + 2xy^2\right)}_{p(x, y)}\d x + \underbrace{\left(x^3 + 2x^2y\right)}_{q(x, y)}\d y = 0
        $$
        Integrabilitätsbedingung:
        $$
            p_y = 3x^2 + 4xy \quad = \quad 3x^2 + 4xy = q_x \quad \checkmark
        $$
        Wir wissen:
        $$
            \underbrace{F = \int q \d y = x^3y + x^2y^2 + c(x)}_{F_y = q} \quad \implies \quad \underbrace{3x^2y + 2xy^2 + c'(x) = 3x^2y + 2xy^2}_{F_x = p} \quad \implies \quad c(x) = c
        $$

        Und insgesamt gilt damit:
        $$
            F = x^3y + x^2y^2 + c
        $$\qed
    \end{solution}

    \newpage
    \question
    Lösen Sie das Anfangswertproblem
    $$
        y' - \sin(x) \cdot y = e^{x-\cos(x)} \quad \text{mit} \quad y(0) = \frac{1}{e}
    $$
    \begin{solution}
        Lösung der homogenen Gleichung:
        $$
            y_h = e^{\int \sin(x) \d x} \quad \iff \quad y_h = ce^{-\cos(x)}
        $$
        Variation der Konstanten:
        $$
            y = c(x) \cdot e^{-\cos(x)} \quad \implies \quad y' = c'(x) \cdot e^{-\cos(x)} + c(x) \cdot \sin(x) \cdot e^{-\cos(x)}
        $$
        Dann gilt:
        $$
            \begin{aligned}
                             & y' - \sin(x) \cdot y                                                                                     &  & = e^{x-\cos(x)}   \\
                \equiv \quad & c'(x) \cdot e^{-\cos(x)} + c(x) \cdot \sin(x) \cdot e^{-\cos(x)} - \sin(x) \cdot c(x) \cdot e^{-\cos(x)} &  & = e^{x-\cos(x)}   \\
                \equiv \quad & c'(x) \cdot e^{-\cos(x)}                                                                                 &  & = e^{x-\cos(x)}   \\
                \equiv \quad & c'(x)                                                                                                    &  & = e^{x}           \\
                \equiv \quad & \int 1 \d c                                                                                              &  & = \int e^{x} \d x \\
                \equiv \quad & c(x) + c_1                                                                                               &  & = e^x + c_2       \\
                \equiv \quad & c(x)                                                                                                     &  & = e^x + \tilde{c}
            \end{aligned}
        $$
        Wir erhalten die allgemeine Lösung der DGL mit:
        $$
            y = \left(e^x + \tilde{c}\right) \cdot e^{-\cos(x)} = ce^{-\cos(x)} + e^{x-\cos(x)}
        $$
        Einsetzen des Anfangswertes:
        $$
            \frac{1}{e} = ce^{-\cos(0)} + e^{0-\cos(0)} \quad \iff \quad \frac{1}{e} = \frac{c}{e} + \frac{1}{e} \quad \iff \quad c = 0
        $$
        Und insgesamt gilt damit:
        $$
            y = e^{x-\cos(x)}
        $$\qed
    \end{solution}

    \newpage
    \question
    Berechnen Sie die Lösung der Differentialgleichung
    $$
        y' - y = g(x)
    $$
    für die Störfunktionen
    \begin{parts}
        \part
        $g(x) = x+1$
        \begin{solution}
            Lösung der homogenen Gleichung:
            $$
                y_h = ce^{\int 1 \d x} = ce^x
            $$
            Bestimmen der partikulären Lösung ($y_p = c_1 x + c_2$):
            $$
                y_p = c_1 x + c_2 \quad \implies \quad y'_p = c_1
            $$

            Dann gilt:
            $$
                \begin{aligned}
                                   & y'_p - y_p                          &  & = x + 1 \\
                    \equiv \quad   & c_1 - (c_1 x + c_2)                 &  & = x + 1 \\
                    \equiv \quad   & (- c_1 )x + (c_1 - c_2)             &  & = x + 1 \\
                    \implies \quad & c_1 = -1 \quad \land \quad c_2 = -2
                \end{aligned}
            $$

            Und insgesamt gilt damit:
            $$
                y = y_h + y_p = ce^x - x - 2
            $$\qed
        \end{solution}

        \part
        $g(x) = e^x$
        \begin{solution}
            Lösung der homogenen Gleichung:
            $$
                y_h = ce^{\int 1 \d x} = ce^x
            $$
            Bestimmen der partikulären Lösung ($y_p = c_1 x e^x$):
            $$
                y_p = c_1 x e^x \quad \implies \quad y'_p =c_1 e^x + c_1 xe^x
            $$

            Dann gilt:
            $$
                \begin{aligned}
                                   & y'_p - y_p                     &  & = e^x \\
                    \equiv \quad   & c_1 e^x + c_1 xe^x - c_1 x e^x &  & = e^x \\
                    \equiv \quad   & c_1 e^x                        &  & = e^x \\
                    \implies \quad & c_1 = 1
                \end{aligned}
            $$

            Und insgesamt gilt damit:
            $$
                y = y_h + y_p = ce^x + x e^x
            $$\qed
        \end{solution}

        \newpage
        \part
        $g(x) = \cos(x)$
        \begin{solution}
            Lösung der homogenen Gleichung:
            $$
                y_h = ce^{\int 1 \d x} = ce^x
            $$
            Bestimmen der partikulären Lösung ($y_p = c_1 \cos(x) + c_2 \sin(x)$):
            $$
                y_p = c_1 \cos(x) + c_2 \sin(x) \quad \implies \quad y'_p =  c_2 \cos(x) -c_1 \sin(x)
            $$

            Dann gilt:
            $$
                \begin{aligned}
                                   & y'_p - y_p                                             &  & = \cos(x) \\
                    \equiv \quad   & c_2 \cos(x) -c_1 \sin(x) - (c_1 \cos(x) + c_2 \sin(x)) &  & = \cos(x) \\
                    \equiv \quad   & (c_2 - c_1) \cos(x) + (c_1 - c_2) \sin(x)              &  & = \cos(x) \\
                    \implies \quad & c_1 = -\frac{1}{2} \quad \land \quad c_2 = \frac{1}{2}
                \end{aligned}
            $$

            Und insgesamt gilt damit:
            $$
                y = y_h + y_p = ce^x + \frac{\sin(x) - \cos(x)}{2}
            $$\qed
        \end{solution}

        \part
        $g(x) = x\cdot e^{-x}$
        \begin{solution}
            Lösung der homogenen Gleichung:
            $$
                y_h = ce^{\int 1 \d x} = ce^x
            $$
            Bestimmen der partikulären Lösung ($y_p = c_1xe^{-x} + c_2e^{-x}$):
            $$
                y_p = c_1xe^{-x} + c_2e^{-x} \quad \implies \quad y'_p = c_1 e^{-x} - c_1 xe^{-x} - c_2e^{-x}
            $$

            Dann gilt:
            $$
                \begin{aligned}
                                   & y'_p - y_p                                                                 &  & = xe^{-x} \\
                    \equiv \quad   & c_1 e^{-x} - c_1 xe^{-x} - c_2e^{-x} - \left(c_1xe^{-x} + c_2e^{-x}\right) &  & = xe^{-x} \\
                    \equiv \quad   & (-2c_1)xe^{-x} + (c_1-2c_2)e^{-x}                                          &  & = xe^{-x} \\
                    \implies \quad & c_1 = -\frac{1}{2} \quad \land \quad c_2 = -\frac{1}{4}
                \end{aligned}
            $$

            Und insgesamt gilt damit:
            $$
                y = y_h + y_p = ce^x - \frac{xe^{-x}}{2} - \frac{e^{-x}}{4}
            $$\qed
        \end{solution}
    \end{parts}

    \newpage
    \question
    Bestimmen Sie die Lösung des folgenden Anfangswertproblems
    $$
        x\cdot y' - y = x^2 \cdot \cos(x) \quad \text{mit} \quad y(\pi) = 2\pi
    $$
    \begin{solution}
        Wir formen um:
        $$
            x\cdot y' - y = x^2 \cdot \cos(x) \quad \iff \quad y' - \frac{1}{x}\cdot y = x \cdot \cos(x)
        $$
        Lösung der homogenen Gleichung:
        $$
            y_h = e^{\int x^{-1} \d x} = e^{\ln \abs{x} + \tilde{c}} = cx
        $$
        Variation der Konstanten:
        $$
            y = c(x) \cdot x \quad \implies \quad y' = c'(x) \cdot x + c(x)
        $$
        Dann gilt:
        $$
            \begin{aligned}
                                                 & y' - \frac{1}{x}\cdot y                              &  & = x \cdot \cos(x)    \\
                \equiv \quad                     & c'(x) \cdot x + c(x) - \frac{1}{x}\cdot c(x) \cdot x &  & = x \cdot \cos(x)    \\
                \equiv \quad                     & c'(x) \cdot x                                        &  & = x \cdot \cos(x)    \\
                \overset{x \neq 0}{\equiv} \quad & c'(x)                                                &  & =\cos(x)             \\
                \equiv \quad                     & \int 1 \d c                                          &  & =\int \cos(x) \d x   \\
                \equiv \quad                     & c(x) + c_1                                           &  & =\sin(x) + c_2       \\
                \equiv \quad                     & c(x)                                                 &  & =\sin(x) + \tilde{c} \\
            \end{aligned}
        $$
        Wir erhalten die allgemeine Lösung der DGL mit:
        $$
            y = (\sin(x) + \tilde{c}) x = cx + x\sin(x)
        $$

        Einsetzen des Anfangswertes:
        $$
            2\pi = c\pi + \pi\sin(\pi) \quad \iff \quad c = 2
        $$
        Und insgesamt gilt damit:
        $$
            y = x(\sin(x) + 2)
        $$\qed
    \end{solution}
\end{questions}
\end{document}